\documentclass[12pt]{article}

\usepackage{amsmath,amssymb}
\usepackage[style=apa,natbib]{biblatex}

\include{preamble.tex}

\addbibresource{references.bib}


\begin{document}

\title{Optimality Conditions and Exact Algorithms for Risk-Averse Bilevel Stochastic Linear Problems}

\author{Bruno M. Pacheco}

\maketitle

\section*{Introduction}

Bilevel stochastic problems can be seen as a generalization of the two-stage problems we have seen in class.
In both cases, there are two decisions to be made: one before and another after the realization of a random variable.
The difference lies in that bilevel stochastic programming does \emph{not} assume that both decisions are made by the same agent.
In turn, this difference leads to a bilevel problem because the two stages do not share the same objective.

The properties of bilevel stochastic linear problems have been studied in the foundational works by \citet{burtscheidtRiskAverseModelsBilevel2020} and \citet{clausExistenceSolutionsClass2022a,clausContinuityRiskaverseBilevel2021}.
The authors consider the more general risk-averse scenario, for which the risk-neutral case becomes a particular instance.
They have presented proofs of the existence of optima and even optimality conditions for (classes of) bilevel problems in which the random variable appears in the right-hand side of the lower level~\citep{burtscheidtRiskAverseModelsBilevel2020}, in the lower level cost function~\citep{clausContinuityRiskaverseBilevel2021} in a quadratic manner, or in both~\citep{clausExistenceSolutionsClass2022a}.
Although those are solid results, their interpretation and applicability is not easy to grasp, as they are proposed for abstract problem classes and assume intricate properties from the components of the mathematical programming models (e.g., constraint functions, solution space, objective function).

The overarching goal of this project is to deeper the understanding of the theoretical results for bilevel stochastic linear problems.
The proposed approach is to explore the implications of these results for two classic textbook examples: the newsvendor problem and the multiproduct assembly problem.
By proposing a bilevel variant of those problems and studying their theoretical properties following \citet{burtscheidtRiskAverseModelsBilevel2020}, I expect to make those results tangible for risk-averse bilevel stochastic linear problems.
Finally, I expect that those applications lead to a clear idea of which exact algorithms can be used to solve the proposed problems, reaching a practical conclusion.



% \bibliographystyle{plain}
\printbibliography
    
\end{document}

