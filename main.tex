\documentclass[12pt]{article}

\usepackage{amsmath,amssymb}
\usepackage[style=apa,natbib]{biblatex}

% Some basic packages
\usepackage[utf8]{inputenc}
\usepackage[T1]{fontenc}
\usepackage{textcomp}
\usepackage[english]{babel}
\usepackage{url}
\usepackage{graphicx}
\usepackage{float}
\usepackage{booktabs}
\usepackage{enumitem}

\pdfminorversion=7

% Don't indent paragraphs, leave some space between them
\usepackage{parskip}

% Hide page number when page is empty
\usepackage{emptypage}
\usepackage{subcaption}
\usepackage{multicol}
\usepackage{xcolor}

% Other font I sometimes use.
% \usepackage{cmbright}

% Math stuff
\usepackage{amsmath, amsfonts, mathtools, amsthm, amssymb}
% Fancy script capitals
\usepackage{mathrsfs}
\usepackage{cancel}
% Bold math
\usepackage{bm}
% Some shortcuts
\newcommand\N{\ensuremath{\mathbb{N}}}
\newcommand\R{\ensuremath{\mathbb{R}}}
\newcommand\Z{\ensuremath{\mathbb{Z}}}
\renewcommand\O{\ensuremath{\emptyset}}
\newcommand\Q{\ensuremath{\mathbb{Q}}}
\newcommand\C{\ensuremath{\mathbb{C}}}
\renewcommand\L{\ensuremath{\mathcal{L}}}

% Easily typeset systems of equations (French package)
\usepackage{systeme}

% Put x \to \infty below \lim
\let\svlim\lim\def\lim{\svlim\limits}

%Make implies and impliedby shorter
\let\implies\Rightarrow
\let\impliedby\Leftarrow
\let\iff\Leftrightarrow
\let\epsilon\varepsilon

% Add \contra symbol to denote contradiction
\usepackage{stmaryrd} % for \lightning
\newcommand\contra{\scalebox{1.5}{$\lightning$}}

% \let\phi\varphi

% Command for short corrections
% Usage: 1+1=\correct{3}{2}

\definecolor{correct}{HTML}{009900}
\newcommand\correct[2]{\ensuremath{\:}{\color{red}{#1}}\ensuremath{\to }{\color{correct}{#2}}\ensuremath{\:}}
\newcommand\green[1]{{\color{correct}{#1}}}

% horizontal rule
\newcommand\hr{
    \noindent\rule[0.5ex]{\linewidth}{0.5pt}
}

% hide parts
\newcommand\hide[1]{}

% si unitx
\usepackage{siunitx}
\sisetup{locale = FR}

% Environments
\makeatother
% For box around Definition, Theorem, \ldots
\usepackage{mdframed}
\mdfsetup{skipabove=1em,skipbelow=0em}
\theoremstyle{definition}
\newmdtheoremenv[nobreak=true]{definitie}{Definitie}
\newmdtheoremenv[nobreak=true]{eigenschap}{Eigenschap}
\newmdtheoremenv[nobreak=true]{gevolg}{Gevolg}
\newmdtheoremenv[nobreak=true]{lemma}{Lemma}
\newmdtheoremenv[nobreak=true]{propositie}{Propositie}
\newmdtheoremenv[nobreak=true]{stelling}{Stelling}
\newmdtheoremenv[nobreak=true]{wet}{Wet}
\newmdtheoremenv[nobreak=true]{postulaat}{Postulaat}
\newmdtheoremenv{conclusie}{Conclusie}
\newmdtheoremenv{toemaatje}{Toemaatje}
\newmdtheoremenv{vermoeden}{Vermoeden}
\newtheorem*{herhaling}{Herhaling}
\newtheorem*{intermezzo}{Intermezzo}
\newtheorem*{notatie}{Notatie}
\newtheorem*{observatie}{Observatie}
\newtheorem*{exe}{Exercise}
\newtheorem*{opmerking}{Opmerking}
\newtheorem*{praktisch}{Praktisch}
\newtheorem*{probleem}{Probleem}
\newtheorem*{terminologie}{Terminologie}
\newtheorem*{toepassing}{Toepassing}
\newtheorem*{uovt}{UOVT}
\newtheorem*{vb}{Voorbeeld}
\newtheorem*{vraag}{Vraag}

\newmdtheoremenv[nobreak=true]{definition}{Definition}
\newmdtheoremenv[nobreak=true]{note}{Note}
\newtheorem*{eg}{Example}
\newtheorem*{notation}{Notation}
\newtheorem*{previouslyseen}{As previously seen}
\newtheorem*{remark}{Remark}
\newtheorem*{problem}{Problem}
\newtheorem*{observe}{Observe}
\newtheorem*{property}{Property}
\newtheorem*{intuition}{Intuition}
\newmdtheoremenv[nobreak=true]{prop}{Proposition}
\newmdtheoremenv[nobreak=true]{theorem}{Theorem}
\newmdtheoremenv[nobreak=true]{corollary}{Corollary}

% End example and intermezzo environments with a small diamond (just like proof
% environments end with a small square)
\usepackage{etoolbox}
\AtEndEnvironment{vb}{\null\hfill$\diamond$}%
\AtEndEnvironment{intermezzo}{\null\hfill$\diamond$}%
% \AtEndEnvironment{opmerking}{\null\hfill$\diamond$}%

% Fix some spacing
% http://tex.stackexchange.com/questions/22119/how-can-i-change-the-spacing-before-theorems-with-amsthm
\makeatletter
\def\thm@space@setup{%
  \thm@preskip=\parskip \thm@postskip=0pt
}


% Exercise 
% Usage:
% \exercise{5}
% \subexercise{1}
% \subexercise{2}
% \subexercise{3}
% gives
% Exercise 5
%   Exercise 5.1
%   Exercise 5.2
%   Exercise 5.3
\newcommand{\exercise}[1]{%
    \def\@exercise{#1}%
    \subsection*{Exercise #1}
}

\newcommand{\subexercise}[1]{%
    \subsubsection*{Exercise \@exercise.#1}
}


% \lecture starts a new lecture (les in dutch)
%
% Usage:
% \lecture{1}{di 12 feb 2019 16:00}{Inleiding}
%
% This adds a section heading with the number / title of the lecture and a
% margin paragraph with the date.

% I use \dateparts here to hide the year (2019). This way, I can easily parse
% the date of each lecture unambiguously while still having a human-friendly
% short format printed to the pdf.

\usepackage{xifthen}
\def\testdateparts#1{\dateparts#1\relax}
\def\dateparts#1 #2 #3 #4 #5\relax{
    \marginpar{\small\textsf{\mbox{#1 #2 #3 #5}}}
}

\def\@lecture{}%
\newcommand{\lecture}[3]{
    \ifthenelse{\isempty{#3}}{%
        \def\@lecture{Lecture #1}%
    }{%
        \def\@lecture{Lecture #1: #3}%
    }%
    \section{\@lecture}
    \marginpar{\small\textsf{\mbox{#2}}}
}



% These are the fancy headers
\usepackage{fancyhdr}
\pagestyle{fancy}

% LE: left even
% RO: right odd
% CE, CO: center even, center odd
% My name for when I print my lecture notes to use for an open book exam.
% \fancyhead[LE,RO]{Gilles Castel}

\fancyhead[RO,LE]{\@lecture} % Right odd,  Left even
\fancyhead[RE,LO]{}          % Right even, Left odd

\fancyfoot[RO,LE]{\thepage}  % Right odd,  Left even
\fancyfoot[RE,LO]{}          % Right even, Left odd
\fancyfoot[C]{\leftmark}     % Center

\makeatother




% Todonotes and inline notes in fancy boxes
\usepackage{todonotes}
\usepackage{tcolorbox}

% Make boxes breakable
\tcbuselibrary{breakable}

% Verbetering is correction in Dutch
% Usage: 
% \begin{verbetering}
%     Lorem ipsum dolor sit amet, consetetur sadipscing elitr, sed diam nonumy eirmod
%     tempor invidunt ut labore et dolore magna aliquyam erat, sed diam voluptua. At
%     vero eos et accusam et justo duo dolores et ea rebum. Stet clita kasd gubergren,
%     no sea takimata sanctus est Lorem ipsum dolor sit amet.
% \end{verbetering}
\newenvironment{verbetering}{\begin{tcolorbox}[
    arc=0mm,
    colback=white,
    colframe=green!60!black,
    title=Opmerking,
    fonttitle=\sffamily,
    breakable
]}{\end{tcolorbox}}

% Noot is note in Dutch. Same as 'verbetering' but color of box is different
\newenvironment{noot}[1]{\begin{tcolorbox}[
    arc=0mm,
    colback=white,
    colframe=white!60!black,
    title=#1,
    fonttitle=\sffamily,
    breakable
]}{\end{tcolorbox}}




% Figure support as explained in my blog post.
\usepackage{import}
\usepackage{xifthen}
\usepackage{pdfpages}
\usepackage{transparent}
\newcommand{\incfig}[1]{%
    \def\svgwidth{\columnwidth}
    \import{./figures/}{#1.pdf_tex}
}

% Fix some stuff
% %http://tex.stackexchange.com/questions/76273/multiple-pdfs-with-page-group-included-in-a-single-page-warning
\pdfsuppresswarningpagegroup=1



\addbibresource{references.bib}


\begin{document}

\title{Optimality Conditions and Exact Algorithms for Risk-Averse Bilevel Stochastic Linear Problems}

\author{Bruno M. Pacheco}

\maketitle


\section*{Introduction}

Bilevel stochastic problems can be seen as a generalization of the two-stage problems we have seen in class.
In both cases, there are two decisions to be made: one before and another after the realization of a random variable.
The difference lies in that bilevel stochastic programming does \emph{not} assume that both decisions are made by the same agent.
In turn, this difference leads to a bilevel problem because the two stages do not share the same objective.

The properties of bilevel stochastic linear problems have been studied in the foundational works by \citet{burtscheidtRiskAverseModelsBilevel2020} and \citet{clausExistenceSolutionsClass2022a,clausContinuityRiskaverseBilevel2021}.
The authors consider the more general risk-averse scenario, for which the risk-neutral case becomes a particular instance.
They have presented proofs of the existence of optima and even optimality conditions for (classes of) bilevel problems in which the random variable appears in the right-hand side of the lower level~\citep{burtscheidtRiskAverseModelsBilevel2020}, in the lower level cost function~\citep{clausContinuityRiskaverseBilevel2021} in a quadratic manner, or in both~\citep{clausExistenceSolutionsClass2022a}.
Although those are solid results, their interpretation and applicability is not easy to grasp, as they are proposed for abstract problem classes and assume intricate properties from the components of the mathematical programming models (e.g., constraint functions, solution space, objective function).

The overarching goal of this project is to deeper the understanding of the theoretical results for bilevel stochastic linear problems.
The proposed approach is to explore the implications of these results for two classic textbook examples: the newsvendor problem and the multiproduct assembly problem.
By proposing a bilevel variant of those problems and studying their theoretical properties following \citet{burtscheidtRiskAverseModelsBilevel2020}, I expect to make those results tangible for risk-averse bilevel stochastic linear problems.
Finally, I expect that those applications lead to a clear idea of which exact algorithms can be used to solve the proposed problems, reaching a practical conclusion.


\section*{The Newsvendor Problem}

As presented in the preliminary report, the newsvendor problem can be formulated as 
\begin{equation}\label{eq:deterministic-2sp-ul}
\begin{split}
    \min_{x} \quad & c x + Q(x,z) \\
    \textrm{s.t.} \quad & 0\le x\le u
,\end{split}
\end{equation}
in which 
\begin{equation}\label{eq:deterministic-2sp-ll}
\begin{split}
    Q(x,z) = \min_{y,w} \quad & -q y - r w \\
    \textrm{s.t.} \quad & y\le z \\
      & y+w \le x \\
      & y,w \ge 0
.\end{split}
\end{equation}
The decision variables $x$, $y$, and $w$ represent, respectively, the amount of newspaper initially bought, the amount of newspaper sold, and the amount of newspaper returned $w$.
The problem is parameterized by the acquisition cost $c$, the newspaper capacity $u$, the demand $z$, the selling price $q$, and the return price $r$.

The traditional two-stage formulation comes from assuming that the demand comes from a random variable $z=Z(\omega)$, where  $\omega$ belongs to a probability space $(\Omega, \mathcal{F},\mathbb{P})$.
Furthermore, it is assumed that the realization of the random variable happens after the first decision (w.r.t. $x$), and before the second decision (w.r.t. $y$ and $w$).
Then, given a risk measure $\mathcal{R}: \mathcal{X} \longrightarrow \R$, where $\mathcal{X}$ is a linear subspace of all $\mathcal{F}$-measurable random variables, the two-stage problem becomes
\begin{equation}
\begin{split}
    \min_{x} \quad & \mathcal{R}[c x + Q(x,Z)] \\
    \textrm{s.t.} \quad & 0\le x\le u
.\end{split}
\end{equation}
Note that if we assume that $\mathcal{R}$ is translation invariant (which is a common property to expect, e.g., variance, value-at-risk), then $\mathcal{R}[c x + Q(x,Z)] = cx + \mathcal{R}[Q(x,Z)]$, which is true, for example, in the risk-neutral case $\mathcal{R} = \mathbb{E}$.

\subsection*{A Bilevel Variant}

In this work, I will assume a slight variation of the original newsvendor problem in which the lower-level decision is made by a different agent, with a different objective.
This may represent, for example, a scenario in which the newspaper acquisition is made by a middle-man, which has different selling and return margins than the newspaper salesperson. 
Instead of \eqref{eq:deterministic-2sp-ul}, we have, then,
\begin{equation}\label{eq:deterministic-bilevel-ul}
\begin{split}
    \min_{x} \quad & f(x,z) = cx + \min\left\{ -q_u y -r_u w : (y,w)\in \Psi(x,z) \right\}  \\
    \textrm{s.t.} \quad & 0\le x\le u
,\end{split}
\end{equation}
in which $\Psi(x,z)$ represents the set of solutions to the lower-level problem, that is,
\begin{equation}\label{eq:newsvendor-ll}
\begin{split}
    \Psi(x,z) = \arg\min_{y,w} \quad & -q_l y - r_l w \\
    \textrm{s.t.} \quad & y\le z \\
      & y+w \le x \\
      & y,w \ge 0
.\end{split}
\end{equation}
Note that the costs differ, that is, the selling and returning costs for the upper level are $q_u$ and $r_u$, resp., while they are $q_l$ and $r_l$ for the lower level.

The first interesting property to analyse, even before introducing the random variable, is that of the function $f$.
\begin{lemma}[\citet{burtscheidtBilevelLinearOptimization2020}, Lemma~17.2.1]\label{lemma:f-lip-cont}
    Function $f$ is real-valued and Lipschitz continuous $\forall x,z \ge 0$.
\end{lemma}
\begin{proof}
    % The dual of \eqref{eq:newsvendor-ll} is 
    % \begin{equation}\label{eq:newsvendor-ll-dual}
    % \begin{split}
    %     \min_{u_z, u_x} \quad & z u_z + x u_x \\
    %     \textrm{s.t.} \quad & u_z + u_x \ge -q_l \\
    %       & u_x \ge -r_l \\
    %       & u_z,u_x \ge 0
    % .\end{split}
    % \end{equation}
    It is easy to see that $\forall x,z \ge 0$, the lower-level problem is feasible, and, thus, a solution to the minimization problem in $f(x,z)$ exists, which renders $f(x,z)$ a real-valued function.
    % Furthermore, the lower-level cost is a continuous function of the parameters (within $x,z \ge 0$)~\citep{pistikopoulosMultiparametricOptimizationControl2021}.

    Now, to demonstrate Lipschitz continuity, take any $x,z,x',z' \ge 0$ such that $f(x,z)\ge f(x',z')$.
    Then, take $(y',w') \in \Psi(x',z')$, which means that $f(x',z') = cx' - q_uy' -r_uw'$, and, thus, for any $(y,w) \in \Psi(x,z)$.
    On top of that, by \citet[Theorem~4.2]{klatteErrorBoundsSolutions1995}, we have that every point $(y',w')\in \Psi(x',z')$ can be expressed as \[
	(y',w') = (y,w) + \Lambda \| (x,z)-(x',z') \| e
    \] for some $(y,w) \in \Psi(x,z)$, a vector $e \in \mathbb{R}^{2}$ with $\|e\|\le 1$, and some constant $\Lambda > 0$.
    Thus, assuming that $c\ge 0$ (which is indeed expected, as it represents a cost),
    \begin{align*}
	|f(x,z)-f(x',z')| &= f(x,z) - cx' + q_uy' + r_uw' \\
			  &\le cx - q_u y - r_u w  - cx' + q_uy' + r_uw' \\
			  &\le  c |x-x'| + \|(q_u,r_u)\| \|(y,w)-(y',w')\| \\
			  &\le  c |x-x'| + \|(q_u,r_u)\| \Lambda \| (x,z)-(x',z') \| \|e\| \\
			  &\le  L_f \| (x,z)-(x',z') \|
    ,\end{align*}
    where $L_f = c + \Lambda \|(q_u,r_u)\|$.
\end{proof}

\subsection*{The Bilevel Stochastic Newsvendor}

As for the two-stage problem, given a random demand $Z$ and a risk measure $\mathcal{R}$, our bilevel newsvendor is interested in solving
\begin{equation}
\begin{split}
    \min_{x} \quad & \mathcal{R}[F(x)] \\
    \textrm{s.t.} \quad & 0\le x\le u
,\end{split}
\end{equation}
where $F(x)=f(x,Z)$ is a random variable parameterized by the upper-level decision $x$.

Before we dive in the properties of the random variable $F(x)$, it is necessary to lay out some definitions about $Z$.
Let $\mu_Z$ be the Borel probability measure induced by $Z$.
This means that, in face of the probability space $\left( \Omega,\mathcal{F},\mathbb{P} \right)$, the probability that $\mu_Z$ associates to a given set $\left\{ z_1,z_2,\ldots \right\} $ of demand values is equal to the probability of the subset of $\Omega$ that contains the respective realization values. 
In other words, if $\left\{ \omega_1,\omega_2,\ldots \right\}\in \Omega$ is such that $z_i=Z(\omega_i)$, for each $i=1,2,\ldots$, then $\mu_Z(\left\{ z_1,z_2,\ldots \right\} = \mathbb{P}\left\{ \omega_1,\omega_2,\ldots \right\}$.
I will write this relationship as $\mu_Z = \mathbb{P} \circ Z^{-1}$, following \citet{burtscheidtBilevelLinearOptimization2020}.

This allows me to reformulate the result that presents the Lipschitz continuity of $F(x)$ for the bilevel newsvendor problem.

\begin{lemma}[\citet{burtscheidtBilevelLinearOptimization2020}, Lemma~17.2.4\footnote{Except the case for probability measures with finite moments of order $p=\infty$.}] 
    If $\mu_Z$ has finite moments of order $p\in [1,\infty)$, then $\exists L>0$ such that \[
	\|F(x) - F(x')\|_p \le L |x - x'|,\quad \forall x,x' \ge 0
    ,\] i.e., $F(x)$ is Lipschitz continuous with respect to the $L^p$-norm $\|F(x)\|_p = \mathbb{E}[|F(x)|^{p}]^{1 / p}$.
\end{lemma}
\begin{proof}
    First, note that, given Lemma~\ref{lemma:f-lip-cont}, $\forall x \ge 0$
    \begin{align*}
	\left( \|F(x)\|_p \right)^{p} &= \int_{\R^+} |f(x,z)|^{p} \mu_Z(dz)  \\
	&= \int_{\R^+} |f(x,z) - f(0,0) + f(0,0)|^{p} \mu_Z(dz)  \\
	&\le  \int_{\R^+} |f(x,z) - f(0,0)|^{p} + |f(0,0)|^{p} \mu_Z(dz)  \\
	&=  |f(0,0)|^{p} + \int_{\R^+} |f(x,z) - f(0,0)|^{p} \mu_Z(dz) \\
	&\le |f(0,0)|^{p} + \int_{\R^+} L_f^{p} \|(x,z)\|^{p} \mu_Z(dz) \\
	&\le |f(0,0)|^{p} + L_f^{p} \int_{\R^+} \|x\|^{p} + \|z\|^{p} \mu_Z(dz) \\
	&= |f(0,0)|^{p} + L_f^{p} \|x\|^{p} + L_f^{p} \int_{\R^+}\|z\|^{p} \mu_Z(dz) < \infty
    ,\end{align*}
    as $\mu_Z$ has finite moments of order $p$.
\end{proof}




% \bibliographystyle{plain}
\printbibliography
    
\end{document}

