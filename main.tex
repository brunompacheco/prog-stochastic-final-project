\documentclass[12pt]{article}

\usepackage{amsmath,amssymb}
\usepackage[style=apa,natbib]{biblatex}

\include{preamble.tex}

\addbibresource{references.bib}


\begin{document}

\title{Optimality Conditions and Exact Algorithms for Risk-Averse Bilevel Stochastic Linear Problems}

\author{Bruno M. Pacheco}

\maketitle


\section*{Introduction}

Bilevel stochastic problems can be seen as a generalization of the two-stage problems we have seen in class.
In both cases, there are two decisions to be made: one before and another after the realization of a random variable.
The difference lies in that bilevel stochastic programming does \emph{not} assume that both decisions are made by the same agent.
In turn, this difference leads to a bilevel problem because the two stages do not share the same objective.

The properties of bilevel stochastic linear problems have been studied in the foundational works by \citet{burtscheidtRiskAverseModelsBilevel2020} and \citet{clausExistenceSolutionsClass2022a,clausContinuityRiskaverseBilevel2021}.
The authors consider the more general risk-averse scenario, for which the risk-neutral case becomes a particular instance.
They have presented proofs of the existence of optima and even optimality conditions for (classes of) bilevel problems in which the random variable appears in the right-hand side of the lower level~\citep{burtscheidtRiskAverseModelsBilevel2020}, in the lower level cost function~\citep{clausContinuityRiskaverseBilevel2021} in a quadratic manner, or in both~\citep{clausExistenceSolutionsClass2022a}.
Although those are solid results, their interpretation and applicability is not easy to grasp, as they are proposed for abstract problem classes and assume intricate properties from the components of the mathematical programming models (e.g., constraint functions, solution space, objective function).

The overarching goal of this project is to deeper the understanding of the theoretical results for bilevel stochastic linear problems.
The proposed approach is to explore the implications of these results for two classic textbook examples: the newsvendor problem and the multiproduct assembly problem.
By proposing a bilevel variant of those problems and studying their theoretical properties following \citet{burtscheidtRiskAverseModelsBilevel2020}, I expect to make those results tangible for risk-averse bilevel stochastic linear problems.
Finally, I expect that those applications lead to a clear idea of which exact algorithms can be used to solve the proposed problems, reaching a practical conclusion.


\section*{The Newsvendor Problem}

As presented in the preliminary report, the newsvendor problem can be formulated as 
\begin{equation}\label{eq:deterministic-2sp-ul}
\begin{split}
    \min_{x} \quad & c x + Q(x,z) \\
    \textrm{s.t.} \quad & 0\le x\le u
,\end{split}
\end{equation}
in which 
\begin{equation}\label{eq:deterministic-2sp-ll}
\begin{split}
    Q(x,z) = \min_{y,w} \quad & -q y - r w \\
    \textrm{s.t.} \quad & y\le z \\
      & y+w \le x \\
      & y,w \ge 0
.\end{split}
\end{equation}
The decision variables $x$, $y$, and $w$ represent, respectively, the amount of newspaper initially bought, the amount of newspaper sold, and the amount of newspaper returned $w$.
The problem is parameterized by the acquisition cost $c$, the newspaper capacity $u$, the demand $z$, the selling price $q$, and the return price $r$.

The traditional two-stage formulation comes from assuming that the demand comes from a random variable $z=Z(\omega)$, where  $\omega$ belongs to a probability space $(\Omega, \mathcal{F},\mathbb{P})$.
Furthermore, it is assumed that the realization of the random variable happens after the first decision (w.r.t. $x$), and before the second decision (w.r.t. $y$ and $w$).
Then, given a risk measure $\mathcal{R}: \mathcal{X} \longrightarrow \R$, where $\mathcal{X}$ is a linear subspace of all $\mathcal{F}$-measurable random variables, the two-stage problem becomes
\begin{equation}\label{eq:two-stage}
\begin{split}
    \min_{x} \quad & \mathcal{R}[c x + Q(x,Z)] \\
    \textrm{s.t.} \quad & 0\le x\le u
.\end{split}
\end{equation}
Note that if we assume that $\mathcal{R}$ is translation invariant, then $\mathcal{R}[c x + Q(x,Z)] = cx + \mathcal{R}[Q(x,Z)]$, which is true in the risk-neutral case $\mathcal{R} = \mathbb{E}$, but also for the value-at-risk and the conditional value-at-risk.

\subsection*{A Bilevel Variant}

In this work, I will assume a slight variation of the original newsvendor problem in which the lower-level decision is made by a different agent, with a different objective.
This may represent, for example, a scenario in which the newspaper acquisition is made by a middle-man, which has different selling and return margins than the newspaper salesperson. 
Instead of \eqref{eq:deterministic-2sp-ul}, we have, then,
\begin{equation}\label{eq:newsvendor-ul}
\begin{split}
    \min_{x} \quad & f(x,z) = cx + \min\left\{ -q_u y -r_u w : (y,w)\in \Psi(x,z) \right\}  \\
    \textrm{s.t.} \quad & 0\le x\le u
,\end{split}
\end{equation}
in which $\Psi(x,z)$ represents the set of solutions to the lower-level problem, that is,
\begin{equation}\label{eq:newsvendor-ll}
\begin{split}
    \Psi(x,z) = \arg\min_{y,w} \quad & -q_l y - r_l w \\
    \textrm{s.t.} \quad & y\le z \\
      & y+w \le x \\
      & y,w \ge 0
.\end{split}
\end{equation}
Note that the costs differ, that is, the selling and returning costs for the upper level are $q_u$ and $r_u$, resp., while they are $q_l$ and $r_l$ for the lower level.

There is, however, an analytic solution to the second stage~\citep{birgeIntroductionStochasticProgramming2011} that assumes the selling price is strictly higher than the return price.
More precisely, assuming $q_l > r_l$, the optimal solution to \eqref{eq:newsvendor-ll} can be written
\begin{align}\label{eq:newsvendor-analytic-solution}
    y^{*}(x,z) &= \min\{x,z\} \\
    w^{*}(x,z) &= \max\{x-z, 0\}
,\end{align}
as long as the problem is feasible, i.e., $\forall x,z \ge 0$.
In other words, $\Psi(x,z)$ is a singleton with  $(\min\{x,z\},\max\{x-z, 0\})$.
Thus, the upper-level cost can be written
\begin{equation}\label{eq:newsvendor-analytic-ul-cost}
    f(x,z) = cx -q_u \min\{x,z\} - r_u \max\{x-z, 0\}
.\end{equation}

The difference arises, for example, if the second stage decision-maker does not have any incentive to sell the newspapers instead of returning them (e.g., no sales commision), in which case $q_l = r_l$ and the optimal solutions are
\begin{equation*}
\begin{split}
    \Psi(x,z) &= \left\{ (y,w)\in \R^2_{+} : y+w = x, y\le z \right\} \\
	      &= \left\{ (y,w)\in \R^2_{+} : y \in \left[ 0,z \right] , w = \max\{x-y,0\} \right\} 
\end{split}
.\end{equation*}
Under the optimistic assumption, the solution that minimizes the upper-level cost is also \eqref{eq:newsvendor-analytic-solution}, as it is a feasible solution that achieves the optimal cost for the lower-level.

Let us consider now the extreme case $q_l < r_l$, for example, the sales commission is inferior to the cost of selling, implying that simply returning the newspaper is the more profitable choice.
Thus, the optimal solution is simply $y^{*}=0$ and $w^{*}=x$, $\forall x,z\ge 0$, which is completely independent of the actual demand.
The upper-level cost becomes $f(x,z) = (c - r_u) x$, which, of course, has an optimal solution at  $x=0$\footnote{That is, assuming $c< r_u$, a purchase cost smaller than the return cost, as the opposite would be nonsensical given the problem's context.}.

At this point, it would be natural to investigate the domain of $f(x,z)$ and its continuity properties.
It is clear in all scenarios that $\text{dom} f = \R^{2}_{+}$, as that renders both lower-level and the upper-level feasible, i.e., the problem has a property similar to relative complete recourse.
For $q_l \ge r_l$, Eq. \eqref{eq:newsvendor-analytic-ul-cost} shows that the upper-level cost function is continuously differentiable almost everywhere with respect to the Lebesgue measure, that is, $f$ has continuous derivatives except at $\left\{ (x,z) \in \R^2_{+} : x = z \right\}$, which has null Lebesgue measure.


\subsection*{The Bilevel Stochastic Newsvendor}

As for the two-stage problem, given a random demand $Z$ and a risk measure $\mathcal{R}$, our bilevel newsvendor is interested in solving
\begin{equation}
\begin{split}
    \min_{x} \quad & \mathcal{R}[F(x)] \\
    \textrm{s.t.} \quad & 0\le x\le u
,\end{split}
\end{equation}
where $F(x)=f(x,Z)$ is a random variable parameterized by the upper-level decision $x$.

I recall that the only case worth investigating here is $q_l \ge r_l$, as otherwise the problem is demand independent, and for that case, the changes from the two-stage variant problem are negligible.
Nevertheless, before we dive in the properties of the random variable $F(x)$ and the cost function $\mathcal{R}[F(x)]$, it is necessary to lay out some definitions about $Z$.

Let $\mu_Z$ be the Borel probability measure induced by $Z$.
This means that, in face of the probability space $\left( \Omega,\mathcal{F},\mathbb{P} \right)$, the probability that $\mu_Z$ associates to a given set $\left\{ z_1,z_2,\ldots \right\} $ of demand values is equal to the probability of the subset of $\Omega$ that contains the respective realization values. 
In other words, if $\left\{ \omega_1,\omega_2,\ldots \right\}\in \Omega$ is such that $z_i=Z(\omega_i)$, for each $i=1,2,\ldots$, then $\mu_Z(\left\{ z_1,z_2,\ldots \right\} = \mathbb{P}\left\{ \omega_1,\omega_2,\ldots \right\}$.
I will write this relationship as $\mu_Z = \mathbb{P} \circ Z^{-1}$, following \citet{burtscheidtBilevelLinearOptimization2020}.
In the following, I will always assume that $\mu_Z$ has finite moments of order $p\in [1,\infty)$, which is equivalent to saying that $Z \in L^{p}\left( \Omega, \mathcal{F},\mathbb{P} \right)$ or \[
    \mathbb{E}[|Z|^{p}] = \int_{\R_{+}} |z|^{p} \mu_Z(dz) < \infty
.\]

It is easy to see that $F(x)$ is Lipschitz continuous.
In fact, $F(x)$ is also convex, as for any realization $Z(\omega)$, \[
    f(x,Z(\omega)) = \begin{cases}
	(c-r)x - (q-r)Z(\omega) &,\, x \ge Z(\omega) \\
	(c-q)x &,\, x < Z(\omega) \\
    \end{cases}
\] and $(c-q) \ge (c-r)$, i.e., the slope of $f(x,Z(\omega))$ increases with $x$ for any $\omega \in \Omega$.
Furthermore, we can see that $\forall x\ge 0$, $\mu_{F(x)}$ has finite moments of order $p\in [1,\infty)$, as $|f(x,z)| \le (c-r)x$, which leads to
\begin{align*}
    \mathbb{E}[|F(x)|^{p}] &= \mathbb{E}[|f(x,Z)|^{p}] = \int_{\R_{+}} |f(x,z)|^{p} \mu_Z(dz) \\
    &\le  (c-r)^{p} x^{p} \int_{\R_{+}} \mu_Z(dz) < \infty
,\end{align*}
thus implying that $F(x) \in L^{p}\left(\Omega, \mathcal{F}, \mathbb{P}\right)$

Let us now assume $\mathcal{R}$ is convex and monotonic.
Then, for any $x,x'>0$, and any $t\in [0,1]$, we use the convexity of $F(x)$ and the monotonicity of $\mathcal{R}$ to see that \[
    \mathcal{R}[F(tx + (1-t)x')] \le \mathcal{R}[tF(x) + (1-t)F(x')]
,\] which, because $\mathcal{R}$ is also convex, shows that $\mathcal{R} \circ F$ is also convex.
Now, by \citep[Proposition~6.5]{shapiroLecturesStochasticProgramming2009}, we know that if $\mathcal{R}$ is convex and monotonic, it will be continuous and subdifferentiable on $L^{p}\left( \Omega, \mathcal{F},\mathbb{P} \right)$.
This, together with the previous result that $F(x)$ is Lipschitz continuous and that its domain ($\R_+$) is compact, leads us to applying \citet[Theorem~6.10]{shapiroLecturesStochasticProgramming2009} to see that $\mathcal{R} \circ F$ is directionally differentiable and its derivative has finite values.
Therefore, gradient-based methods are exact algorithms for this problem and will (likely) return globally optimal solutions.

Of course, the above results are unnecessary for the risk-neutral case, as one can much more easily derive analytical solutions (cf. \citet[Chapter~1.e]{birgeIntroductionStochasticProgramming2011} and \citet[Chapter~1.2.1]{shapiroLecturesStochasticProgramming2009}).
On the other hand, the results are perfectly suitable for the case of the Conditional Value-at-Risk, which is, in fact, a coherent risk measure.
For the Value-at-Risk, one can use \citet[Theorem~2]{ivanovBilevelStochasticLinear2014} to show that it is also a continuous risk measure on the image of $F(x)$, and, thus, a solution to the problem exists.
To the best of my knowledge, there is no result ensuring differentiability of the Value-at-Risk with respect to the problem variable $x$.
However, we can see that
\begin{align*}
    \mathbb{P}[ f(x,Z) \le \eta] &= \int_{\R_{+}} 1[f(x,z) \le \eta] \mu_Z(dz) \\
    &= \int_{0}^{x} 1[f(x,z) \le \eta] \mu_Z(dz) + \int_{x}^{\infty} 1[f(x,z) \le \eta] \mu_Z(dz) \\
    &= \int_{0}^{x} 1[(c-r)x - (q-r)z \le \eta] \mu_Z(dz) + \int_{x}^{\infty} 1[(c-q)x \le \eta] \mu_Z(dz) \\
    &= \int_{0}^{x} 1[z \ge ((c-r)x - \eta) / (q-r)] \mu_Z(dz) + 1[(c-q)x \le \eta] \int_{x}^{\infty} \mu_Z(dz) \\
    &= 1[(c-q)x \le \eta] \int_{\max\{((c-r)x - \eta) / (q-r),0\}}^{x} \mu_Z(dz) + 1[(c-q)x \le \eta] \int_{x}^{\infty} \mu_Z(dz) \\
    &= 1[(c-q)x \le \eta] \int_{\max\{((c-r)x - \eta) / (q-r),0\}}^{\infty} \mu_Z(dz) \\
    &= \begin{cases}
	1 &,\, \eta \ge (c-r)x  \\
	\int_{((c-r)x - \eta) / (q-r)}^{\infty} \mu_Z(dz)  &,\, (c-q)x \le \eta \le  (c-r)x \\
	0  &,\, \eta \le (c-q)x
    \end{cases}
.\end{align*}
Thus, for $\alpha \in (0,1]$, we have 
\begin{align*}
    \text{VaR}_{\alpha}[F(x)] &= \inf_{\eta}\left\{ \eta \in \R : \mathbb{P}[ F(x) \le \eta] \ge \alpha  \right\} \\
    &= \inf_{\eta}\left\{ \eta \in \R : \mathbb{P}[ f(x,Z) \le \eta] \ge \alpha  \right\} \\
    &= \inf_{\eta}\left\{ \eta \in [(c-q)x,(c-r)x] : \int_{((c-r)x - \eta) / (q-r)}^{\infty} \mu_Z(dz) \ge \alpha   \right\}
,\end{align*} 
which, if we assume $Z$ has a smooth CDF, implies that $\text{VaR}_\alpha \circ F$ is differentiable through a simple application of the implicit function theorem\footnote{This is due to $\text{VaR}[F(x)]$ being formulated as a minimization (as the infimum is achieved) that is equivalent to solving $\int_{((c-r)x - \eta) / (q-r)}^{\infty} \mu_Z(dz) - \alpha = 0$.}.


\section*{The Multiproduct Assembly Problem}

As in the preliminary report\footnote{In the preliminary report, I used $\bm{d}$ to refer to the problem parameter, which I here decided to use $\bm{z}$ for consistency with respect to the newsvendor problem.}, the multiproduct assembly problem is, here, as
\begin{align*}
    \min_{\bm{x}} \quad & f(\bm{x},\bm{z}) = \bm{c}^{T} \bm{x} + Q(\bm{x}, \bm{z}) \\
    \textrm{s.t.} \quad & 0 \le \bm{x} \le \bm{u}
,\end{align*}
in which
\begin{align*}
    Q(\bm{x},\bm{z}) = \min_{\bm{y},\bm{w}} \quad & -\bm{q}^{T} \bm{y} - \bm{r}^{T}\bm{w} \\
    \textrm{s.t.} \quad & A^{T}\bm{y} + \bm{w} \le \bm{x} \\
      & \bm{y} \le \bm{z} \\
      & \bm{y}, \bm{w} \ge 0
.\end{align*}
Note here we have \emph{vector} decision variables $\bm{x}=(x_1,\ldots,x_m)$, $\bm{y}=(y_1,\ldots,y_{n})$, and $\bm{w}=(w_1,\ldots,w_m)$.
Each $x_{j}$ and $w_j$ represent the amount of each part $j$ that was bought and that remained, resp., while each $y_{i}$ represents the amount of product $i$ that was assembled.
The external parameters $\bm{z}=(z_1,\ldots,z_n)$ represent the demand for each of the $n$ products that can be assembled.
Finally, at the first stage,  $\bm{c}$ and $\bm{u}$ are the acquisition cost and the storage capacity for the parts, and at the second stage $\bm{q}$ and $\bm{r}$ are the sales price of the products and the recycling price of the parts, while in $A=[a_{ij}]$, $a_{ij}$ is the amount of part $j$ required to build product $i$.

As for the newsvendor problem, if we assume an uncertain demand modeled as the realization of a random variable $\bm{z} = Z(\omega)$, where $\omega$ belongs to a probability space $\left( \Omega, \mathcal{F}, \mathbb{P} \right)$, the two-stage problem is formulated very similarly to \eqref{eq:two-stage}
\begin{equation}\label{eq:two-stage-ndim}
\begin{split}
    \min_{\bm{x}} \quad & \mathcal{R}[\bm{c}^{T}\bm{x} + Q(\bm{x},Z)] \\
    \textrm{s.t.} \quad & 0\le \bm{x}\le \bm{u}
.\end{split}
\end{equation}
However, I highlight that, despite using the same notation, for the multiproduct assembly problem $Z(\omega)\in \R^{n}_{+}$.

\subsection*{A Bilevel Variant}

Once again, the consideration that the second-stage decision is made by a different agent is used to yield a bilevel problem.
In here, this may represent, for example, the dynamics between a supplier (first stage) and a manufacturer (second stage).
The bilevel variant can be written
\begin{equation}\label{eq:multiproduct-ul}
\begin{split}
    \min_{\bm{x}} \quad & f(\bm{x},\bm{z}) = \bm{c}^{T}\bm{x} + \min\left\{ -\bm{q}^{T}_u \bm{y} -\bm{r}^{T}_u \bm{w} : (\bm{y},\bm{w})\in \Psi(\bm{x},\bm{z}) \right\}  \\
    \textrm{s.t.} \quad & 0\le \bm{x}\le \bm{u}
,\end{split}
\end{equation}
in which
\begin{equation}\label{eq:multiproduct-ll}
\begin{split}
    \Psi(\bm{x},\bm{z}) = \arg\min_{\bm{y},\bm{w}} \quad & -\bm{q}_l \bm{y} - \bm{r}_l \bm{w} \\
    \textrm{s.t.} \quad & A^{T}\bm{y}+\bm{w} \le \bm{x} \\
      & \bm{y}\le \bm{z} \\
      & \bm{y},\bm{w} \ge 0
.\end{split}
\end{equation}

For the multiproduct assembly problem there is no analytic solution such as the one for the newsvendor problem.
Therefore, the first interesting property to analyse on the bilevel variant, even before introducing the random variable, is that of the continuity of function $f$.
\begin{lemma}[\citet{burtscheidtBilevelLinearOptimization2020}, Lemma~17.2.1]\label{lemma:f-lip-cont}
    Function $f$ is real-valued and Lipschitz continuous on $\R^{m+n}_{+}$.
\end{lemma}
\begin{proof}
    % The dual of \eqref{eq:newsvendor-ll} is 
    % \begin{equation}\label{eq:newsvendor-ll-dual}
    % \begin{split}
    %     \min_{u_z, u_x} \quad & z u_z + x u_x \\
    %     \textrm{s.t.} \quad & u_z + u_x \ge -q_l \\
    %       & u_x \ge -r_l \\
    %       & u_z,u_x \ge 0
    % .\end{split}
    % \end{equation}
    It is easy to see that $\forall \left( \bm{x},\bm{z} \right) \in \R^{m+n}_{+}$ the lower-level problem is feasible, e.g., through the naïve assignment $(\overline{\bm{y}},\overline{\bm{w}})=(0,\bm{x})$.
    Also, the lower-level is also bounded, as $\bm{x}$ and $\bm{z}$ are, resp., upper-bounds for $\bm{w}$ and $\bm{y}$.
    As a consequence, the minimization problem in $f(\bm{x},\bm{z})$ exists (i.e., is well-defined), which renders $f(\bm{x},\bm{z})$ a real-valued function, or, equivalently, $\text{dom} f = \R^{m+n}_{+}$.
    % Furthermore, the lower-level cost is a continuous function of the parameters (within $x,z \ge 0$)~\citep{pistikopoulosMultiparametricOptimizationControl2021}.

    Now, to demonstrate Lipschitz continuity, take any $(\bm{x},\bm{z}),(\bm{x}',\bm{z}') \in \R^{m+n}_{+}$ and assume, without loss of generalization, that $f(\bm{x},\bm{z})\ge f(\bm{x}',\bm{z}')$.
    Then, take $(\bm{y}',\bm{w}') \in \Psi(\bm{x}',\bm{z}')$ such that $f(\bm{x}',\bm{z}') =  \bm{c}^{T}\bm{x}' - \bm{q}_u \bm{y}' -\bm{r}_u \bm{w}'$.
    Parametric programming theory \citep[Lemma~4.1]{klatteErrorBoundsSolutions1995} gives us that for every point $(\bm{y}',\bm{w}')\in \Psi(\bm{x}',\bm{z}')$, there exists $\Lambda > 0$ such that for any other point $(\bm{y},\bm{w})\in \Psi(\bm{x},\bm{z})$ \[
	(\bm{y}',\bm{w}') = (\bm{y},\bm{w}) + \Lambda \| (\bm{x},\bm{z})-(\bm{x}',\bm{z}') \| \bm{e}
    \] for some a vector $\bm{e} \in \mathbb{R}^{m+n}$ with $\|\bm{e}\|\le 1$.
    % Thus, assuming that $c\ge 0$ (which is indeed expected, as it represents a cost),, and, thus,
    Thus, for any $(\bm{y},\bm{w}) \in \Psi(\bm{x},\bm{z})$,
    \begin{align*}
	|f(\bm{x},\bm{z})-f(\bm{x}',\bm{z}')| &= f(\bm{x},\bm{z}) - \bm{c}^{T}\bm{x}' + \bm{q}_u^{T} \bm{y}' + \bm{r}_u^{T} \bm{w}' \\
			  &\le \bm{c}^{T} \bm{x} - \bm{q}_u^{T} \bm{y} - \bm{r}_u^{T} \bm{w}  - \bm{c}^{T}\bm{x}' + \bm{q}_u^{T} \bm{y}' + \bm{r}_u^{T} \bm{w}' \\
			  &\le  \|\bm{c}\| \|\bm{x}-\bm{x}'\| + \|(\bm{q}_u,\bm{r}_u)\| \|(\bm{y},\bm{w})-(\bm{y}',\bm{w}')\| \\
			  &\le  \|\bm{c}\| \|\bm{x}-\bm{x}'\| + \|(\bm{q}_u,\bm{r}_u)\| \Lambda\|(\bm{x},\bm{z})-(\bm{x}',\bm{z}')\| \|\bm{e}\| \\
			  &\le  L_f \|(\bm{x},\bm{z})-(\bm{x}',\bm{z}')\|
    ,\end{align*}
    where $L_f = \|\bm{c}\| + \Lambda \|(\bm{q}_u,\bm{r}_u)\|$.
\end{proof}

\subsection*{The Bilevel Stochastic Multiproduct Assembly}

Knowing that the deterministic problem is solvable let's us wander into the stochastic version of the problem.
Recall that it is assumed here, just as for the two-stage variant, that the demand is the realization of the random variable $Z$, which happens prior to the lower-level decision.
Furthermore, I consider here a risk-averse problem, such that, given a risk measure $\mathcal{R}$ as previously, the problem of interest is
\begin{equation}\label{eq:multiprod-bis-prob}
\begin{split}
    \min_{\bm{x}} \quad & \mathcal{R}[F(\bm{x})] \\
    \textrm{s.t.} \quad & 0\le \bm{x}\le \bm{u}
,\end{split}
\end{equation}
where $F(\bm{x})=f(\bm{x},Z)$ is a random variable parameterized by the upper-level decision $\bm{x}$.
Note that $F(\bm{x})$ is well-defined as any upper-level decision $\bm{x}\in \R^{m}_{+}$ renders the lower-level feasible.

As for the newsvendor problem, let $\mu_Z$ be the Borel probability measure induced by Z.
I will assume that $\mu_Z$ has finite moments of order $p\in [1,\infty)$, i.e., $Z \in L^{p}(\Omega,\mathcal{F},\mathbb{P})$.
This allows me to state the Lipschitz continuity of the random variable $F(\bm{x})$.
\begin{lemma}[\citet{burtscheidtBilevelLinearOptimization2020}, Lemma~17.2.4\footnote{Except the case for probability measures with finite moments of order $p=\infty$.}]\label{lemma:F-lip-cont}
    If $\mu_Z$ has finite moments of order $p\in [1,\infty)$, then $\exists L>0$ such that \[
	\|F(\bm{x}) - F(\bm{x}')\|_p \le L \|\bm{x} - \bm{x}'\|,\quad \forall \bm{x},\bm{x}' \in \R^{m}_{+}
    ,\] i.e., $F(\bm{x})$ is Lipschitz continuous with respect to the $L^p$-norm $\|F(\bm{x})\|_p = \mathbb{E}[|F(\bm{x})|^{p}]^{1 / p}$.
\end{lemma}
\begin{proof}
    First, note that, given Lemma~\ref{lemma:f-lip-cont}, $\forall \bm{x} \in \R^{m}_{+}$
    \begin{align*}
	\left( \|F(\bm{x})\|_p \right)^{p} &= \int_{\R^{n}_{+}} |f(\bm{x},\bm{z})|^{p} \mu_Z(d\bm{z})  \\
	&= \int_{\R^{n}_{+}} |f(\bm{x},\bm{z}) - f(0,0) + f(0,0)|^{p} \mu_Z(d\bm{z})  \\
	&\le  \int_{\R^{n}_{+}} |f(0,0)|^{p} + |f(\bm{x},\bm{z}) - f(0,0)|^{p} \mu_Z(d\bm{z})  \\
	&\le |f(0,0)|^{p} + \int_{\R^{n}_{+}} L_f^{p} \|(\bm{x},\bm{z})\|^{p} \mu_Z(d\bm{z}) \\
	&\le |f(0,0)|^{p} + L_f^{p} \int_{\R^{n}_{+}} \|\bm{x}\|^{p} + \|\bm{z}\|^{p} \mu_Z(d\bm{z}) \\
	&= |f(0,0)|^{p} + L_f^{p} \|\bm{x}\|^{p} + L_f^{p} \int_{\R^{n}_{+}}\|\bm{z}\|^{p} \mu_Z(d\bm{z}) < \infty
    ,\end{align*}
    as $\mu_Z$ has finite moments of order $p$.
    Then, for any $\bm{x},\bm{x}' \in \R^{m}_{+}$,
    \begin{align*}
        \|F(\bm{x}) - F(\bm{x}')\|_{p} &= \left( \int_{\R^{n}_{+}} |f(\bm{x},\bm{z}) - f(\bm{x}',\bm{z})|^{p}\mu_Z(d\bm{z}) \right)^{1 / p} \\ 
	&\le \left( \int_{\R^{n}_{+}} \left( L_f \|(\bm{x},\bm{z}) - (\bm{x}',\bm{z})\| \right)^{p}\mu_Z(d\bm{z}) \right)^{1 / p} =  L_f \|\bm{x}-\bm{x}'\|
    .\end{align*}
\end{proof}

\subsubsection*{Continuity and Existence of Solutions}

Let us assume once again that $\mathcal{R}$ is convex and monotonic.
Then, because $L^{p}(\Omega,\mathcal{F},\mathbb{P})$ is a Banach lattice under the $L^{p}$-norm, we can use \citet[Theorem~4.1]{cheriditoRiskMeasuresOrlicz2009} to see that $\mathcal{R}$ is locally Lipschitz continuous.
Thus, by Lemma~\ref{lemma:F-lip-cont}, $\mathcal{R} \circ F$ is also locally Lipschitz continuous on $\R^{m}_{+}$, which grants us that \eqref{eq:multiprod-bis-prob} is solvable, as any feasible solution has finite cost\footnote{Note that the set of feasible solutions is a compact subset of the domain of $F$.}.
I will make these results tangible for the three risk measures chosen.

\begin{lemma}[\citet{burtscheidtRiskAverseModelsBilevel2019}, Theorem~3.4]\label{eq:lip-cont-E}
    If $\mu_Z$ has finite moments of order $p=1$, then $\mathcal{R}_{\mathbb{E}} = \mathbb{E}$ is real-valued and Lipschitz continuous.
\end{lemma}
\begin{proof}
    This proof is very similar to the proof of Lemma~\ref{lemma:F-lip-cont}.
    First, for any $\bm{x}\in \R^{n}_+$,
    \begin{align*}
	|\mathcal{R}_{\mathbb{E}}[F(\bm{x})]| &= |\mathbb{E}[f(\bm{x},Z)]| \le \int_{\R^{n}_+} |f(\bm{x},\bm{z})| \mu_Z(d\bm{z}) \\
	&\le |f(0,0)| + L_f \| \bm{x} \| + L_f \int_{\R^{n}_+} \|\bm{z}\| \mu_Z(d\bm{z}) < \infty
    .\end{align*}

    Now, for any $\bm{x},\bm{x}'\in \R^{m}_+$,
    \begin{align*}
	|\mathcal{R}_\mathbb{E}[F(\bm{x})] - \mathcal{R}_\mathbb{E}[F(\bm{x}')]| &= \left| \int_{\R^{n}_+} \left( f(\bm{x},\bm{z}) - f(\bm{x}',\bm{z}) \right) \mu_Z(d\bm{z}) \right|  \\
	&\le \int_{\R^{n}_+} |f(\bm{x},\bm{z}) - f(\bm{x}',\bm{z})| \mu_Z(d\bm{z}) \\
	&= L_f \|\bm{x} - \bm{x}'\|
    .\end{align*}
\end{proof}

The case for the conditional value-at-risk is more intricate.
Let \[
    \text{CVaR}_\alpha[Y] = \min_{t\in \R} t + \frac{1}{1-\alpha} \mathbb{E}[\max\{Y-t,0\}]
\] for any $Y \in L^{1}(\Omega,\mathcal{F},\mathbb{P})$.
Then,
\begin{align*}
    \mathcal{R}_{\text{CVaR}_\alpha}[F(\bm{x})] &= \text{CVaR}_\alpha[F(\bm{x})] = \min_{t\in \R} t + \frac{1}{1-\alpha} \mathbb{E}[\max\{F(\bm{x})-t,0\}] \\
    &= \min_{t} t + \frac{1}{1-\alpha} \int_{\R^{n}_+} \max\{f(\bm{x},\bm{z})-t,0\} \mu_Z(d\bm{z}) \\
    &= \min_{t, g(\bm{z})} t + \frac{1}{1-\alpha} \int_{\R^{n}_+} g(\bm{z}) \mu_Z(d\bm{z}) \\
    &\quad\quad \text{s.t.} \quad g(Z) + t \ge f(\bm{x},Z),\, \text{a.s.} \\
    &\quad\quad \quad \quad g(Z) \ge 0,\, \text{a.s.}
,\end{align*}
that is, the properties of $\mathcal{R}_{\text{CVaR}_\alpha}$ can be shown from the study of the above parametric optimization problem.
Note that the above problem is linear and that the parameter appears only on the right-hand side of the constraints.
Thus, if $Z$ has finite support, \citep[Theorem~2.1]{pistikopoulosMultiparametricOptimizationControl2021} shows that the objective is continuous, convex, and piecewise affine with respect to $f(\bm{x},\overline{\bm{z}}),\,\forall \overline{\bm{z}}\in \text{supp}(\mu_Z)$.
Because $f$ is Lipschitz continuous on $\bm{x}$, then $\mathcal{R}_{\text{CVaR}_\alpha} \circ F$ is also Lipschitz continuous.
Intuitively, note that as one smoothly varies $\bm{x}$, $f(\bm{x},Z)$ also changes smoothly, which smoothly contracts or expands the polyhedron that defines the set of feasible solutions to the problem, which, in turn, implies that the optimal objective value varies smoothly with respect to $\bm{x}$.
% In fact, because $f(\bm{x},Z)$ is on the right-hand side of the constraints, the contraction/expansion happens such that each facet moves orthogonally to its surface.

If $Z$ has infinite support, the problem becomes an optimization problem over a functional space (i.e., $L^{1}(\Omega,\mathcal{F},\mathbb{P})$).
Although the intuition still holds, to the best of my knowledge no one has shown that the problem

\textcolor{red}{\textbf{TODO}}

If $Z$ has finite support, then the 
Therefore, specifically showing that $\mathcal{R}_{\text{CVaR}_\alpha}$ is real-valued is equivalent to saying that the above optimization problem over a function space is solvable, which has been shown in \citet[Chapter~6.2.4]{shapiroLecturesStochasticProgramming2009}.
In fact, it is shown that indeed the conditional value-at-risk is convex and monotonic and, thus, falls into the category of risk measures discussed earlier.

Finally, the value-at-risk, which is \emph{not} convex, needs a special treatment.






% \bibliographystyle{plain}
\printbibliography
    
\end{document}

